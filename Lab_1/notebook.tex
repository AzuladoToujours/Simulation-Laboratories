
% Default to the notebook output style

    


% Inherit from the specified cell style.




    
\documentclass[11pt]{article}

    
    
    \usepackage[T1]{fontenc}
    % Nicer default font (+ math font) than Computer Modern for most use cases
    \usepackage{mathpazo}

    % Basic figure setup, for now with no caption control since it's done
    % automatically by Pandoc (which extracts ![](path) syntax from Markdown).
    \usepackage{graphicx}
    % We will generate all images so they have a width \maxwidth. This means
    % that they will get their normal width if they fit onto the page, but
    % are scaled down if they would overflow the margins.
    \makeatletter
    \def\maxwidth{\ifdim\Gin@nat@width>\linewidth\linewidth
    \else\Gin@nat@width\fi}
    \makeatother
    \let\Oldincludegraphics\includegraphics
    % Set max figure width to be 80% of text width, for now hardcoded.
    \renewcommand{\includegraphics}[1]{\Oldincludegraphics[width=.8\maxwidth]{#1}}
    % Ensure that by default, figures have no caption (until we provide a
    % proper Figure object with a Caption API and a way to capture that
    % in the conversion process - todo).
    \usepackage{caption}
    \DeclareCaptionLabelFormat{nolabel}{}
    \captionsetup{labelformat=nolabel}

    \usepackage{adjustbox} % Used to constrain images to a maximum size 
    \usepackage{xcolor} % Allow colors to be defined
    \usepackage{enumerate} % Needed for markdown enumerations to work
    \usepackage{geometry} % Used to adjust the document margins
    \usepackage{amsmath} % Equations
    \usepackage{amssymb} % Equations
    \usepackage{textcomp} % defines textquotesingle
    % Hack from http://tex.stackexchange.com/a/47451/13684:
    \AtBeginDocument{%
        \def\PYZsq{\textquotesingle}% Upright quotes in Pygmentized code
    }
    \usepackage{upquote} % Upright quotes for verbatim code
    \usepackage{eurosym} % defines \euro
    \usepackage[mathletters]{ucs} % Extended unicode (utf-8) support
    \usepackage[utf8x]{inputenc} % Allow utf-8 characters in the tex document
    \usepackage{fancyvrb} % verbatim replacement that allows latex
    \usepackage{grffile} % extends the file name processing of package graphics 
                         % to support a larger range 
    % The hyperref package gives us a pdf with properly built
    % internal navigation ('pdf bookmarks' for the table of contents,
    % internal cross-reference links, web links for URLs, etc.)
    \usepackage{hyperref}
    \usepackage{longtable} % longtable support required by pandoc >1.10
    \usepackage{booktabs}  % table support for pandoc > 1.12.2
    \usepackage[inline]{enumitem} % IRkernel/repr support (it uses the enumerate* environment)
    \usepackage[normalem]{ulem} % ulem is needed to support strikethroughs (\sout)
                                % normalem makes italics be italics, not underlines
    

    
    
    % Colors for the hyperref package
    \definecolor{urlcolor}{rgb}{0,.145,.698}
    \definecolor{linkcolor}{rgb}{.71,0.21,0.01}
    \definecolor{citecolor}{rgb}{.12,.54,.11}

    % ANSI colors
    \definecolor{ansi-black}{HTML}{3E424D}
    \definecolor{ansi-black-intense}{HTML}{282C36}
    \definecolor{ansi-red}{HTML}{E75C58}
    \definecolor{ansi-red-intense}{HTML}{B22B31}
    \definecolor{ansi-green}{HTML}{00A250}
    \definecolor{ansi-green-intense}{HTML}{007427}
    \definecolor{ansi-yellow}{HTML}{DDB62B}
    \definecolor{ansi-yellow-intense}{HTML}{B27D12}
    \definecolor{ansi-blue}{HTML}{208FFB}
    \definecolor{ansi-blue-intense}{HTML}{0065CA}
    \definecolor{ansi-magenta}{HTML}{D160C4}
    \definecolor{ansi-magenta-intense}{HTML}{A03196}
    \definecolor{ansi-cyan}{HTML}{60C6C8}
    \definecolor{ansi-cyan-intense}{HTML}{258F8F}
    \definecolor{ansi-white}{HTML}{C5C1B4}
    \definecolor{ansi-white-intense}{HTML}{A1A6B2}

    % commands and environments needed by pandoc snippets
    % extracted from the output of `pandoc -s`
    \providecommand{\tightlist}{%
      \setlength{\itemsep}{0pt}\setlength{\parskip}{0pt}}
    \DefineVerbatimEnvironment{Highlighting}{Verbatim}{commandchars=\\\{\}}
    % Add ',fontsize=\small' for more characters per line
    \newenvironment{Shaded}{}{}
    \newcommand{\KeywordTok}[1]{\textcolor[rgb]{0.00,0.44,0.13}{\textbf{{#1}}}}
    \newcommand{\DataTypeTok}[1]{\textcolor[rgb]{0.56,0.13,0.00}{{#1}}}
    \newcommand{\DecValTok}[1]{\textcolor[rgb]{0.25,0.63,0.44}{{#1}}}
    \newcommand{\BaseNTok}[1]{\textcolor[rgb]{0.25,0.63,0.44}{{#1}}}
    \newcommand{\FloatTok}[1]{\textcolor[rgb]{0.25,0.63,0.44}{{#1}}}
    \newcommand{\CharTok}[1]{\textcolor[rgb]{0.25,0.44,0.63}{{#1}}}
    \newcommand{\StringTok}[1]{\textcolor[rgb]{0.25,0.44,0.63}{{#1}}}
    \newcommand{\CommentTok}[1]{\textcolor[rgb]{0.38,0.63,0.69}{\textit{{#1}}}}
    \newcommand{\OtherTok}[1]{\textcolor[rgb]{0.00,0.44,0.13}{{#1}}}
    \newcommand{\AlertTok}[1]{\textcolor[rgb]{1.00,0.00,0.00}{\textbf{{#1}}}}
    \newcommand{\FunctionTok}[1]{\textcolor[rgb]{0.02,0.16,0.49}{{#1}}}
    \newcommand{\RegionMarkerTok}[1]{{#1}}
    \newcommand{\ErrorTok}[1]{\textcolor[rgb]{1.00,0.00,0.00}{\textbf{{#1}}}}
    \newcommand{\NormalTok}[1]{{#1}}
    
    % Additional commands for more recent versions of Pandoc
    \newcommand{\ConstantTok}[1]{\textcolor[rgb]{0.53,0.00,0.00}{{#1}}}
    \newcommand{\SpecialCharTok}[1]{\textcolor[rgb]{0.25,0.44,0.63}{{#1}}}
    \newcommand{\VerbatimStringTok}[1]{\textcolor[rgb]{0.25,0.44,0.63}{{#1}}}
    \newcommand{\SpecialStringTok}[1]{\textcolor[rgb]{0.73,0.40,0.53}{{#1}}}
    \newcommand{\ImportTok}[1]{{#1}}
    \newcommand{\DocumentationTok}[1]{\textcolor[rgb]{0.73,0.13,0.13}{\textit{{#1}}}}
    \newcommand{\AnnotationTok}[1]{\textcolor[rgb]{0.38,0.63,0.69}{\textbf{\textit{{#1}}}}}
    \newcommand{\CommentVarTok}[1]{\textcolor[rgb]{0.38,0.63,0.69}{\textbf{\textit{{#1}}}}}
    \newcommand{\VariableTok}[1]{\textcolor[rgb]{0.10,0.09,0.49}{{#1}}}
    \newcommand{\ControlFlowTok}[1]{\textcolor[rgb]{0.00,0.44,0.13}{\textbf{{#1}}}}
    \newcommand{\OperatorTok}[1]{\textcolor[rgb]{0.40,0.40,0.40}{{#1}}}
    \newcommand{\BuiltInTok}[1]{{#1}}
    \newcommand{\ExtensionTok}[1]{{#1}}
    \newcommand{\PreprocessorTok}[1]{\textcolor[rgb]{0.74,0.48,0.00}{{#1}}}
    \newcommand{\AttributeTok}[1]{\textcolor[rgb]{0.49,0.56,0.16}{{#1}}}
    \newcommand{\InformationTok}[1]{\textcolor[rgb]{0.38,0.63,0.69}{\textbf{\textit{{#1}}}}}
    \newcommand{\WarningTok}[1]{\textcolor[rgb]{0.38,0.63,0.69}{\textbf{\textit{{#1}}}}}
    
    
    % Define a nice break command that doesn't care if a line doesn't already
    % exist.
    \def\br{\hspace*{\fill} \\* }
    % Math Jax compatability definitions
    \def\gt{>}
    \def\lt{<}
    % Document parameters
    \title{Lab\_1\_Parte\_2\_Funciones\_Gausianas}
    
    
    

    % Pygments definitions
    
\makeatletter
\def\PY@reset{\let\PY@it=\relax \let\PY@bf=\relax%
    \let\PY@ul=\relax \let\PY@tc=\relax%
    \let\PY@bc=\relax \let\PY@ff=\relax}
\def\PY@tok#1{\csname PY@tok@#1\endcsname}
\def\PY@toks#1+{\ifx\relax#1\empty\else%
    \PY@tok{#1}\expandafter\PY@toks\fi}
\def\PY@do#1{\PY@bc{\PY@tc{\PY@ul{%
    \PY@it{\PY@bf{\PY@ff{#1}}}}}}}
\def\PY#1#2{\PY@reset\PY@toks#1+\relax+\PY@do{#2}}

\expandafter\def\csname PY@tok@w\endcsname{\def\PY@tc##1{\textcolor[rgb]{0.73,0.73,0.73}{##1}}}
\expandafter\def\csname PY@tok@c\endcsname{\let\PY@it=\textit\def\PY@tc##1{\textcolor[rgb]{0.25,0.50,0.50}{##1}}}
\expandafter\def\csname PY@tok@cp\endcsname{\def\PY@tc##1{\textcolor[rgb]{0.74,0.48,0.00}{##1}}}
\expandafter\def\csname PY@tok@k\endcsname{\let\PY@bf=\textbf\def\PY@tc##1{\textcolor[rgb]{0.00,0.50,0.00}{##1}}}
\expandafter\def\csname PY@tok@kp\endcsname{\def\PY@tc##1{\textcolor[rgb]{0.00,0.50,0.00}{##1}}}
\expandafter\def\csname PY@tok@kt\endcsname{\def\PY@tc##1{\textcolor[rgb]{0.69,0.00,0.25}{##1}}}
\expandafter\def\csname PY@tok@o\endcsname{\def\PY@tc##1{\textcolor[rgb]{0.40,0.40,0.40}{##1}}}
\expandafter\def\csname PY@tok@ow\endcsname{\let\PY@bf=\textbf\def\PY@tc##1{\textcolor[rgb]{0.67,0.13,1.00}{##1}}}
\expandafter\def\csname PY@tok@nb\endcsname{\def\PY@tc##1{\textcolor[rgb]{0.00,0.50,0.00}{##1}}}
\expandafter\def\csname PY@tok@nf\endcsname{\def\PY@tc##1{\textcolor[rgb]{0.00,0.00,1.00}{##1}}}
\expandafter\def\csname PY@tok@nc\endcsname{\let\PY@bf=\textbf\def\PY@tc##1{\textcolor[rgb]{0.00,0.00,1.00}{##1}}}
\expandafter\def\csname PY@tok@nn\endcsname{\let\PY@bf=\textbf\def\PY@tc##1{\textcolor[rgb]{0.00,0.00,1.00}{##1}}}
\expandafter\def\csname PY@tok@ne\endcsname{\let\PY@bf=\textbf\def\PY@tc##1{\textcolor[rgb]{0.82,0.25,0.23}{##1}}}
\expandafter\def\csname PY@tok@nv\endcsname{\def\PY@tc##1{\textcolor[rgb]{0.10,0.09,0.49}{##1}}}
\expandafter\def\csname PY@tok@no\endcsname{\def\PY@tc##1{\textcolor[rgb]{0.53,0.00,0.00}{##1}}}
\expandafter\def\csname PY@tok@nl\endcsname{\def\PY@tc##1{\textcolor[rgb]{0.63,0.63,0.00}{##1}}}
\expandafter\def\csname PY@tok@ni\endcsname{\let\PY@bf=\textbf\def\PY@tc##1{\textcolor[rgb]{0.60,0.60,0.60}{##1}}}
\expandafter\def\csname PY@tok@na\endcsname{\def\PY@tc##1{\textcolor[rgb]{0.49,0.56,0.16}{##1}}}
\expandafter\def\csname PY@tok@nt\endcsname{\let\PY@bf=\textbf\def\PY@tc##1{\textcolor[rgb]{0.00,0.50,0.00}{##1}}}
\expandafter\def\csname PY@tok@nd\endcsname{\def\PY@tc##1{\textcolor[rgb]{0.67,0.13,1.00}{##1}}}
\expandafter\def\csname PY@tok@s\endcsname{\def\PY@tc##1{\textcolor[rgb]{0.73,0.13,0.13}{##1}}}
\expandafter\def\csname PY@tok@sd\endcsname{\let\PY@it=\textit\def\PY@tc##1{\textcolor[rgb]{0.73,0.13,0.13}{##1}}}
\expandafter\def\csname PY@tok@si\endcsname{\let\PY@bf=\textbf\def\PY@tc##1{\textcolor[rgb]{0.73,0.40,0.53}{##1}}}
\expandafter\def\csname PY@tok@se\endcsname{\let\PY@bf=\textbf\def\PY@tc##1{\textcolor[rgb]{0.73,0.40,0.13}{##1}}}
\expandafter\def\csname PY@tok@sr\endcsname{\def\PY@tc##1{\textcolor[rgb]{0.73,0.40,0.53}{##1}}}
\expandafter\def\csname PY@tok@ss\endcsname{\def\PY@tc##1{\textcolor[rgb]{0.10,0.09,0.49}{##1}}}
\expandafter\def\csname PY@tok@sx\endcsname{\def\PY@tc##1{\textcolor[rgb]{0.00,0.50,0.00}{##1}}}
\expandafter\def\csname PY@tok@m\endcsname{\def\PY@tc##1{\textcolor[rgb]{0.40,0.40,0.40}{##1}}}
\expandafter\def\csname PY@tok@gh\endcsname{\let\PY@bf=\textbf\def\PY@tc##1{\textcolor[rgb]{0.00,0.00,0.50}{##1}}}
\expandafter\def\csname PY@tok@gu\endcsname{\let\PY@bf=\textbf\def\PY@tc##1{\textcolor[rgb]{0.50,0.00,0.50}{##1}}}
\expandafter\def\csname PY@tok@gd\endcsname{\def\PY@tc##1{\textcolor[rgb]{0.63,0.00,0.00}{##1}}}
\expandafter\def\csname PY@tok@gi\endcsname{\def\PY@tc##1{\textcolor[rgb]{0.00,0.63,0.00}{##1}}}
\expandafter\def\csname PY@tok@gr\endcsname{\def\PY@tc##1{\textcolor[rgb]{1.00,0.00,0.00}{##1}}}
\expandafter\def\csname PY@tok@ge\endcsname{\let\PY@it=\textit}
\expandafter\def\csname PY@tok@gs\endcsname{\let\PY@bf=\textbf}
\expandafter\def\csname PY@tok@gp\endcsname{\let\PY@bf=\textbf\def\PY@tc##1{\textcolor[rgb]{0.00,0.00,0.50}{##1}}}
\expandafter\def\csname PY@tok@go\endcsname{\def\PY@tc##1{\textcolor[rgb]{0.53,0.53,0.53}{##1}}}
\expandafter\def\csname PY@tok@gt\endcsname{\def\PY@tc##1{\textcolor[rgb]{0.00,0.27,0.87}{##1}}}
\expandafter\def\csname PY@tok@err\endcsname{\def\PY@bc##1{\setlength{\fboxsep}{0pt}\fcolorbox[rgb]{1.00,0.00,0.00}{1,1,1}{\strut ##1}}}
\expandafter\def\csname PY@tok@kc\endcsname{\let\PY@bf=\textbf\def\PY@tc##1{\textcolor[rgb]{0.00,0.50,0.00}{##1}}}
\expandafter\def\csname PY@tok@kd\endcsname{\let\PY@bf=\textbf\def\PY@tc##1{\textcolor[rgb]{0.00,0.50,0.00}{##1}}}
\expandafter\def\csname PY@tok@kn\endcsname{\let\PY@bf=\textbf\def\PY@tc##1{\textcolor[rgb]{0.00,0.50,0.00}{##1}}}
\expandafter\def\csname PY@tok@kr\endcsname{\let\PY@bf=\textbf\def\PY@tc##1{\textcolor[rgb]{0.00,0.50,0.00}{##1}}}
\expandafter\def\csname PY@tok@bp\endcsname{\def\PY@tc##1{\textcolor[rgb]{0.00,0.50,0.00}{##1}}}
\expandafter\def\csname PY@tok@fm\endcsname{\def\PY@tc##1{\textcolor[rgb]{0.00,0.00,1.00}{##1}}}
\expandafter\def\csname PY@tok@vc\endcsname{\def\PY@tc##1{\textcolor[rgb]{0.10,0.09,0.49}{##1}}}
\expandafter\def\csname PY@tok@vg\endcsname{\def\PY@tc##1{\textcolor[rgb]{0.10,0.09,0.49}{##1}}}
\expandafter\def\csname PY@tok@vi\endcsname{\def\PY@tc##1{\textcolor[rgb]{0.10,0.09,0.49}{##1}}}
\expandafter\def\csname PY@tok@vm\endcsname{\def\PY@tc##1{\textcolor[rgb]{0.10,0.09,0.49}{##1}}}
\expandafter\def\csname PY@tok@sa\endcsname{\def\PY@tc##1{\textcolor[rgb]{0.73,0.13,0.13}{##1}}}
\expandafter\def\csname PY@tok@sb\endcsname{\def\PY@tc##1{\textcolor[rgb]{0.73,0.13,0.13}{##1}}}
\expandafter\def\csname PY@tok@sc\endcsname{\def\PY@tc##1{\textcolor[rgb]{0.73,0.13,0.13}{##1}}}
\expandafter\def\csname PY@tok@dl\endcsname{\def\PY@tc##1{\textcolor[rgb]{0.73,0.13,0.13}{##1}}}
\expandafter\def\csname PY@tok@s2\endcsname{\def\PY@tc##1{\textcolor[rgb]{0.73,0.13,0.13}{##1}}}
\expandafter\def\csname PY@tok@sh\endcsname{\def\PY@tc##1{\textcolor[rgb]{0.73,0.13,0.13}{##1}}}
\expandafter\def\csname PY@tok@s1\endcsname{\def\PY@tc##1{\textcolor[rgb]{0.73,0.13,0.13}{##1}}}
\expandafter\def\csname PY@tok@mb\endcsname{\def\PY@tc##1{\textcolor[rgb]{0.40,0.40,0.40}{##1}}}
\expandafter\def\csname PY@tok@mf\endcsname{\def\PY@tc##1{\textcolor[rgb]{0.40,0.40,0.40}{##1}}}
\expandafter\def\csname PY@tok@mh\endcsname{\def\PY@tc##1{\textcolor[rgb]{0.40,0.40,0.40}{##1}}}
\expandafter\def\csname PY@tok@mi\endcsname{\def\PY@tc##1{\textcolor[rgb]{0.40,0.40,0.40}{##1}}}
\expandafter\def\csname PY@tok@il\endcsname{\def\PY@tc##1{\textcolor[rgb]{0.40,0.40,0.40}{##1}}}
\expandafter\def\csname PY@tok@mo\endcsname{\def\PY@tc##1{\textcolor[rgb]{0.40,0.40,0.40}{##1}}}
\expandafter\def\csname PY@tok@ch\endcsname{\let\PY@it=\textit\def\PY@tc##1{\textcolor[rgb]{0.25,0.50,0.50}{##1}}}
\expandafter\def\csname PY@tok@cm\endcsname{\let\PY@it=\textit\def\PY@tc##1{\textcolor[rgb]{0.25,0.50,0.50}{##1}}}
\expandafter\def\csname PY@tok@cpf\endcsname{\let\PY@it=\textit\def\PY@tc##1{\textcolor[rgb]{0.25,0.50,0.50}{##1}}}
\expandafter\def\csname PY@tok@c1\endcsname{\let\PY@it=\textit\def\PY@tc##1{\textcolor[rgb]{0.25,0.50,0.50}{##1}}}
\expandafter\def\csname PY@tok@cs\endcsname{\let\PY@it=\textit\def\PY@tc##1{\textcolor[rgb]{0.25,0.50,0.50}{##1}}}

\def\PYZbs{\char`\\}
\def\PYZus{\char`\_}
\def\PYZob{\char`\{}
\def\PYZcb{\char`\}}
\def\PYZca{\char`\^}
\def\PYZam{\char`\&}
\def\PYZlt{\char`\<}
\def\PYZgt{\char`\>}
\def\PYZsh{\char`\#}
\def\PYZpc{\char`\%}
\def\PYZdl{\char`\$}
\def\PYZhy{\char`\-}
\def\PYZsq{\char`\'}
\def\PYZdq{\char`\"}
\def\PYZti{\char`\~}
% for compatibility with earlier versions
\def\PYZat{@}
\def\PYZlb{[}
\def\PYZrb{]}
\makeatother


    % Exact colors from NB
    \definecolor{incolor}{rgb}{0.0, 0.0, 0.5}
    \definecolor{outcolor}{rgb}{0.545, 0.0, 0.0}



    
    % Prevent overflowing lines due to hard-to-break entities
    \sloppy 
    % Setup hyperref package
    \hypersetup{
      breaklinks=true,  % so long urls are correctly broken across lines
      colorlinks=true,
      urlcolor=urlcolor,
      linkcolor=linkcolor,
      citecolor=citecolor,
      }
    % Slightly bigger margins than the latex defaults
    
    \geometry{verbose,tmargin=1in,bmargin=1in,lmargin=1in,rmargin=1in}
    
    

    \begin{document}
    
    
    \maketitle
    
    

    
    \subsection{Laboratorio 1 - Parte 2}\label{laboratorio-1---parte-2}

\subsubsection{Regresión logística y Funciones Discriminantes
Gausianas}\label{regresiuxf3n-loguxedstica-y-funciones-discriminantes-gausianas}

\subsubsection{2019 - I}\label{i}

\paragraph{Profesor: Julián D. Arias
Londoño}\label{profesor-juliuxe1n-d.-arias-londouxf1o}

\paragraph{julian.ariasl@udea.edu.co}\label{julian.ariasludea.edu.co}

    \subsection{Guía del laboratorio}\label{guuxeda-del-laboratorio}

En este archivo va a encontrar tanto celdas de código cómo celdas de
texto con las instrucciones para desarrollar el laboratorio.

Lea atentamente las instrucciones entregadas en las celdas de texto
correspondientes y proceda con la solución de las preguntas planteadas.

Nota: No olvide ir ejecutando las celdas de código de arriba hacia abajo
para que no tenga errores de importación de librerías o por falta de
definición de variables.

    \begin{Verbatim}[commandchars=\\\{\}]
{\color{incolor}In [{\color{incolor}1}]:} \PY{k+kn}{from} \PY{n+nn}{\PYZus{}\PYZus{}future\PYZus{}\PYZus{}} \PY{k}{import} \PY{n}{division}
        \PY{o}{\PYZpc{}}\PY{k}{matplotlib} inline
        \PY{k+kn}{import} \PY{n+nn}{numpy} \PY{k}{as} \PY{n+nn}{np}
        \PY{k+kn}{import} \PY{n+nn}{matplotlib}\PY{n+nn}{.}\PY{n+nn}{pyplot} \PY{k}{as} \PY{n+nn}{plt}
        \PY{k+kn}{from} \PY{n+nn}{scipy} \PY{k}{import} \PY{n}{stats}
        
        \PY{c+c1}{\PYZsh{}Algunas advertencias que queremos evitar}
        \PY{k+kn}{import} \PY{n+nn}{warnings}
        \PY{n}{warnings}\PY{o}{.}\PY{n}{filterwarnings}\PY{p}{(}\PY{l+s+s2}{\PYZdq{}}\PY{l+s+s2}{always}\PY{l+s+s2}{\PYZdq{}}\PY{p}{)}
\end{Verbatim}


    \begin{Verbatim}[commandchars=\\\{\}]
{\color{incolor}In [{\color{incolor} }]:} \PY{c+c1}{\PYZsh{}Tomado de https://github.com/rramosp/mooc\PYZhy{}grader}
        \PY{k+kn}{from} \PY{n+nn}{Autentication} \PY{k}{import} \PY{o}{*}\PY{c+c1}{\PYZsh{}python 3}
        
        \PY{k+kn}{import} \PY{n+nn}{inspect}\PY{o}{,} \PY{n+nn}{urllib}
        \PY{n}{html}\PY{p}{,} \PY{n}{auth\PYZus{}code}\PY{p}{,} \PY{n}{userinfo} \PY{o}{=} \PY{n}{google\PYZus{}authenticate}\PY{p}{(}\PY{n}{PORT\PYZus{}NUMBER}\PY{o}{=}\PY{l+m+mi}{8080}\PY{p}{)}
        \PY{n}{html}
\end{Verbatim}


    \subsection{Ejercicio 1: Contextualización del
problema}\label{ejercicio-1-contextualizaciuxf3n-del-problema}

El problema de \emph{clasificación} que están cargados en el archivo
\texttt{DatosClases.mat}. Las variables o caracterísicas son guardadas
en X y la variable de salida es guardada en la variable Y.

Responda las siguientes preguntas y grafique los datos usando la función
\href{https://matplotlib.org/gallery/shapes_and_collections/scatter.html}{scatter}
de matplotlib y responda las siguientes preguntas

    \begin{Verbatim}[commandchars=\\\{\}]
{\color{incolor}In [{\color{incolor}90}]:} \PY{k+kn}{import} \PY{n+nn}{scipy}\PY{n+nn}{.}\PY{n+nn}{io}
         \PY{n}{mat} \PY{o}{=} \PY{n}{scipy}\PY{o}{.}\PY{n}{io}\PY{o}{.}\PY{n}{loadmat}\PY{p}{(}\PY{l+s+s1}{\PYZsq{}}\PY{l+s+s1}{DB/DatosClases.mat}\PY{l+s+s1}{\PYZsq{}}\PY{p}{)}
         \PY{n}{X} \PY{o}{=} \PY{n}{mat}\PY{p}{[}\PY{l+s+s1}{\PYZsq{}}\PY{l+s+s1}{X}\PY{l+s+s1}{\PYZsq{}}\PY{p}{]} \PY{c+c1}{\PYZsh{} Matriz X de muestras con las características}
         \PY{n}{Y} \PY{o}{=} \PY{n}{mat}\PY{p}{[}\PY{l+s+s1}{\PYZsq{}}\PY{l+s+s1}{Y}\PY{l+s+s1}{\PYZsq{}}\PY{p}{]} \PY{c+c1}{\PYZsh{} Variable de salida}
\end{Verbatim}


    1.1 ¿Cuántas clases tiene el problema?:

    \begin{Verbatim}[commandchars=\\\{\}]
{\color{incolor}In [{\color{incolor}91}]:} \PY{c+c1}{\PYZsh{}unique retorna un vector con los valores unicos de el array que se introduce por parametro}
         \PY{n}{N\PYZus{}CLASS} \PY{o}{=} \PY{n}{np}\PY{o}{.}\PY{n}{unique}\PY{p}{(}\PY{n}{Y}\PY{p}{)} 
         \PY{n}{N\PYZus{}CLASS} \PY{o}{=} \PY{n}{np}\PY{o}{.}\PY{n}{size}\PY{p}{(}\PY{n}{N\PYZus{}CLASS}\PY{p}{)}
         \PY{n}{N\PYZus{}CLASS}
\end{Verbatim}


\begin{Verbatim}[commandchars=\\\{\}]
{\color{outcolor}Out[{\color{outcolor}91}]:} 2
\end{Verbatim}
            
    1.2 ¿Cuántas características tiene el problema?:

    \begin{Verbatim}[commandchars=\\\{\}]
{\color{incolor}In [{\color{incolor}92}]:} \PY{n}{N\PYZus{}FEATURES} \PY{o}{=} \PY{n}{np}\PY{o}{.}\PY{n}{size}\PY{p}{(}\PY{n}{X}\PY{p}{,}\PY{l+m+mi}{1}\PY{p}{)}
         \PY{n}{N\PYZus{}FEATURES}
\end{Verbatim}


\begin{Verbatim}[commandchars=\\\{\}]
{\color{outcolor}Out[{\color{outcolor}92}]:} 2
\end{Verbatim}
            
    1.3 ¿Cuántas muestras tiene el problema?:

    \begin{Verbatim}[commandchars=\\\{\}]
{\color{incolor}In [{\color{incolor}93}]:} \PY{n}{N\PYZus{}SAMPLES} \PY{o}{=} \PY{n}{np}\PY{o}{.}\PY{n}{size}\PY{p}{(}\PY{n}{X}\PY{p}{,}\PY{l+m+mi}{0}\PY{p}{)}
         \PY{n}{N\PYZus{}SAMPLES}
\end{Verbatim}


\begin{Verbatim}[commandchars=\\\{\}]
{\color{outcolor}Out[{\color{outcolor}93}]:} 500
\end{Verbatim}
            
    1.4 ¿El problema es linealmente separable?:

    \textbf{R//:} Si, ya que como se aprecia en la figura 1, podemos trazar
una linea que separe las dos clases.

    \begin{Verbatim}[commandchars=\\\{\}]
{\color{incolor}In [{\color{incolor}94}]:} \PY{n}{X1\PYZus{}Square} \PY{o}{=} \PY{n}{X}\PY{p}{[}\PY{p}{:}\PY{p}{,}\PY{l+m+mi}{0}\PY{p}{]}\PY{o}{*}\PY{o}{*}\PY{l+m+mi}{2}
         \PY{n}{X2\PYZus{}Square} \PY{o}{=} \PY{n}{X}\PY{p}{[}\PY{p}{:}\PY{p}{,}\PY{l+m+mi}{1}\PY{p}{]}\PY{o}{*}\PY{o}{*}\PY{l+m+mi}{2}
         \PY{n}{plt}\PY{o}{.}\PY{n}{scatter}\PY{p}{(}\PY{n}{X1\PYZus{}Square}\PY{p}{,}\PY{n}{X2\PYZus{}Square}\PY{p}{,}\PY{n}{c}\PY{o}{=}\PY{n}{Y}\PY{p}{[}\PY{p}{:}\PY{p}{,}\PY{l+m+mi}{0}\PY{p}{]}\PY{p}{)}
         \PY{n}{plt}\PY{o}{.}\PY{n}{title}\PY{p}{(}\PY{l+s+s1}{\PYZsq{}}\PY{l+s+s1}{Figura 1: Caracteristicas al cuadrado}\PY{l+s+s1}{\PYZsq{}}\PY{p}{)}
         \PY{n}{plt}\PY{o}{.}\PY{n}{xlabel}\PY{p}{(}\PY{l+s+s1}{\PYZsq{}}\PY{l+s+s1}{Caracteristica X1\PYZca{}2}\PY{l+s+s1}{\PYZsq{}}\PY{p}{)}
         \PY{n}{plt}\PY{o}{.}\PY{n}{ylabel}\PY{p}{(}\PY{l+s+s1}{\PYZsq{}}\PY{l+s+s1}{Caracteristica X2\PYZca{}2}\PY{l+s+s1}{\PYZsq{}}\PY{p}{)}
         \PY{n}{plt}\PY{o}{.}\PY{n}{show}\PY{p}{(}\PY{p}{)}
\end{Verbatim}


    \begin{center}
    \adjustimage{max size={0.9\linewidth}{0.9\paperheight}}{output_14_0.png}
    \end{center}
    { \hspace*{\fill} \\}
    
    Gráfica del conjunto de datos

    \begin{Verbatim}[commandchars=\\\{\}]
{\color{incolor}In [{\color{incolor}95}]:} \PY{n}{ALL} \PY{o}{=} \PY{n}{np}\PY{o}{.}\PY{n}{concatenate}\PY{p}{(}\PY{p}{(}\PY{n}{X}\PY{p}{,}\PY{n}{Y}\PY{p}{)}\PY{p}{,} \PY{n}{axis}\PY{o}{=}\PY{l+m+mi}{1}\PY{p}{)}
         \PY{n}{CLASS1} \PY{o}{=} \PY{n}{ALL}\PY{p}{[}\PY{n}{np}\PY{o}{.}\PY{n}{where}\PY{p}{(}\PY{n}{ALL}\PY{p}{[}\PY{p}{:}\PY{p}{,}\PY{o}{\PYZhy{}}\PY{l+m+mi}{1}\PY{p}{]} \PY{o}{==} \PY{l+m+mi}{0}\PY{p}{)}\PY{p}{]}
         \PY{n}{CLASS2} \PY{o}{=} \PY{n}{ALL}\PY{p}{[}\PY{n}{np}\PY{o}{.}\PY{n}{where}\PY{p}{(}\PY{n}{ALL}\PY{p}{[}\PY{p}{:}\PY{p}{,}\PY{o}{\PYZhy{}}\PY{l+m+mi}{1}\PY{p}{]} \PY{o}{==} \PY{l+m+mi}{1}\PY{p}{)}\PY{p}{]}
         
         \PY{n}{plt}\PY{o}{.}\PY{n}{scatter}\PY{p}{(}\PY{n}{CLASS1}\PY{p}{[}\PY{p}{:}\PY{p}{,}\PY{l+m+mi}{0}\PY{p}{]}\PY{p}{,}\PY{n}{CLASS1}\PY{p}{[}\PY{p}{:}\PY{p}{,}\PY{l+m+mi}{1}\PY{p}{]}\PY{p}{,}\PY{n}{c}\PY{o}{=}\PY{l+s+s1}{\PYZsq{}}\PY{l+s+s1}{yellow}\PY{l+s+s1}{\PYZsq{}}\PY{p}{)}
         \PY{n}{plt}\PY{o}{.}\PY{n}{scatter}\PY{p}{(}\PY{n}{CLASS2}\PY{p}{[}\PY{p}{:}\PY{p}{,}\PY{l+m+mi}{0}\PY{p}{]}\PY{p}{,}\PY{n}{CLASS2}\PY{p}{[}\PY{p}{:}\PY{p}{,}\PY{l+m+mi}{1}\PY{p}{]}\PY{p}{,}\PY{n}{c}\PY{o}{=}\PY{l+s+s1}{\PYZsq{}}\PY{l+s+s1}{black}\PY{l+s+s1}{\PYZsq{}}\PY{p}{)}
         \PY{n}{plt}\PY{o}{.}\PY{n}{title}\PY{p}{(}\PY{l+s+s1}{\PYZsq{}}\PY{l+s+s1}{Figura 2: Caracteristicas}\PY{l+s+s1}{\PYZsq{}}\PY{p}{)}
         \PY{n}{plt}\PY{o}{.}\PY{n}{xlabel}\PY{p}{(}\PY{l+s+s1}{\PYZsq{}}\PY{l+s+s1}{X1}\PY{l+s+s1}{\PYZsq{}}\PY{p}{)}
         \PY{n}{plt}\PY{o}{.}\PY{n}{ylabel}\PY{p}{(}\PY{l+s+s1}{\PYZsq{}}\PY{l+s+s1}{X2}\PY{l+s+s1}{\PYZsq{}}\PY{p}{)}
         \PY{n}{plt}\PY{o}{.}\PY{n}{show}\PY{p}{(}\PY{p}{)}
\end{Verbatim}


    \begin{center}
    \adjustimage{max size={0.9\linewidth}{0.9\paperheight}}{output_16_0.png}
    \end{center}
    { \hspace*{\fill} \\}
    
    \subsection{Ejercicio 2: Completar
código}\label{ejercicio-2-completar-cuxf3digo}

En este laboratorio se va a realizar un procedimiento análogo al
laboratorio anterior, pero con el modelo de \emph{regresión logística}
que sirve para resolver problemas de clasificación (en principio
biclase).

Analice los siguientes métodos a la luz de la teoría vista para el
modelo de regresión logística, tales como la función de activación
(sigmoidal), el modelo de regresión logística (logistic\_regression),
potencia del polinomio y el cálculo del error en clasificación
(error\_logistic) y el gradiente descendente.

Una vez comprenda su funcionamiento proceda a realizar lo siguiente:

\begin{enumerate}
\def\labelenumi{\arabic{enumi}.}
\tightlist
\item
  Completar el código del método de gradiente\_descedente\_logistic con
  la regla de actualización de los parámetros para el problema de
  clasificación
\end{enumerate}

\[w_j(iter) = w_j(iter-1) - \eta \frac{\partial E(w)}{\partial w_j}\]

Nota: Para el problema de clasificación tenga presente que si ya
implementó la regla de actualización de parámetros para el modelo de
regresión polinomial múltiple, este punto es trivial, puesto que sólo
tiene que incluir la función sigmoidal tal como lo vimos en la teoría.

\begin{enumerate}
\def\labelenumi{\arabic{enumi}.}
\setcounter{enumi}{1}
\tightlist
\item
  Graficar el error de clasificación durante las iteraciones del
  algoritmo. La gráfica debe llevar título y los correspondientes
  nombres de los ejes.
\end{enumerate}

Nota: Observe que el método logistic\_regression ya hace el llamado a la
función sigmoidal

    \begin{Verbatim}[commandchars=\\\{\}]
{\color{incolor}In [{\color{incolor}127}]:} \PY{c+c1}{\PYZsh{}Función de activación Sigmoidal}
          \PY{k}{def} \PY{n+nf}{sigmoidal}\PY{p}{(}\PY{n}{z}\PY{p}{,} \PY{n}{showPlot}\PY{p}{)}\PY{p}{:}
              
              \PY{c+c1}{\PYZsh{}Complete la siguiente línea con el código para calcular la salida de la función sigmoidal}
              \PY{n}{s} \PY{o}{=} \PY{l+m+mi}{1}\PY{o}{/}\PY{p}{(}\PY{l+m+mi}{1}\PY{o}{+}\PY{n}{np}\PY{o}{.}\PY{n}{exp}\PY{p}{(}\PY{o}{\PYZhy{}}\PY{l+m+mi}{1}\PY{o}{*}\PY{n}{z}\PY{p}{)}\PY{p}{)}
              
              \PY{c+c1}{\PYZsh{}Complete el código para realizar la gráfica de la función aquí}
              \PY{k}{if} \PY{n}{showPlot} \PY{o}{==} \PY{k+kc}{True}\PY{p}{:}
                  \PY{n}{plt}\PY{o}{.}\PY{n}{plot}\PY{p}{(}\PY{n}{z}\PY{p}{,}\PY{n}{s}\PY{p}{,}\PY{l+s+s1}{\PYZsq{}}\PY{l+s+s1}{r+}\PY{l+s+s1}{\PYZsq{}}\PY{p}{)}
                  \PY{n}{plt}\PY{o}{.}\PY{n}{title}\PY{p}{(}\PY{l+s+s1}{\PYZsq{}}\PY{l+s+s1}{Figura 3: Función sigmoidal}\PY{l+s+s1}{\PYZsq{}}\PY{p}{)}
                  \PY{n}{plt}\PY{o}{.}\PY{n}{show}\PY{p}{(}\PY{p}{)}
              \PY{k}{return} \PY{n}{s}
          
          
          \PY{c+c1}{\PYZsh{}Modelo Regresión logística}
          \PY{k}{def} \PY{n+nf}{logistic\PYZus{}regression}\PY{p}{(}\PY{n}{X}\PY{p}{,} \PY{n}{W}\PY{p}{,} \PY{n}{showPlot}\PY{o}{=}\PY{k+kc}{False}\PY{p}{)}\PY{p}{:}
              \PY{c+c1}{\PYZsh{}Con np.dot se realiza el producto matricial. Aquí X (extendida) tiene dim [Nxd] y W es dim [dx1]}
              \PY{n}{Yest} \PY{o}{=} \PY{n}{np}\PY{o}{.}\PY{n}{dot}\PY{p}{(}\PY{n}{X}\PY{p}{,}\PY{n}{W}\PY{p}{)}
              
              \PY{n}{Y\PYZus{}lest} \PY{o}{=} \PY{n}{sigmoidal}\PY{p}{(}\PY{n}{Yest}\PY{p}{,} \PY{n}{showPlot}\PY{p}{)}
              
              \PY{c+c1}{\PYZsh{}Se asignan los valores a 1 o 0 según el modelo de regresión logística definido}
              \PY{n}{pos} \PY{o}{=} \PY{l+m+mi}{0}
              \PY{k}{for} \PY{n}{tag} \PY{o+ow}{in} \PY{n}{Y\PYZus{}lest}\PY{p}{:}
                  
                  \PY{k}{if} \PY{n}{tag} \PY{o}{\PYZgt{}} \PY{l+m+mf}{0.5}\PY{p}{:}
                      \PY{n}{Y\PYZus{}lest}\PY{p}{[}\PY{n}{pos}\PY{p}{]} \PY{o}{=} \PY{l+m+mi}{1}
                  \PY{k}{elif} \PY{n}{tag} \PY{o}{\PYZlt{}} \PY{l+m+mf}{0.5}\PY{p}{:}
                      \PY{n}{Y\PYZus{}lest}\PY{p}{[}\PY{n}{pos}\PY{p}{]} \PY{o}{=} \PY{l+m+mi}{0}
                  
                  \PY{n}{pos} \PY{o}{+}\PY{o}{=} \PY{l+m+mi}{1}
              
              \PY{k}{return} \PY{n}{Y\PYZus{}lest}    \PY{c+c1}{\PYZsh{}Y estimado: Esta variable contiene ya tiene la salida de sigm(f(X,W))}
          
          
          \PY{c+c1}{\PYZsh{}En este laboratorio solo trabajaremos el caso lineal (grado 1), pero se pueden probar otras fronteras}
          \PY{k}{def} \PY{n+nf}{potenciaPolinomio}\PY{p}{(}\PY{n}{X}\PY{p}{,}\PY{n}{grado}\PY{p}{)}\PY{p}{:}
              \PY{n}{X2} \PY{o}{=} \PY{n}{X}
              
              \PY{k}{if} \PY{n}{grado} \PY{o}{!=} \PY{l+m+mi}{1}\PY{p}{:}
                  \PY{k}{for} \PY{n}{i} \PY{o+ow}{in} \PY{n+nb}{range}\PY{p}{(}\PY{l+m+mi}{2}\PY{p}{,}\PY{n}{grado}\PY{o}{+}\PY{l+m+mi}{1}\PY{p}{)}\PY{p}{:}
                      \PY{n}{Xadd} \PY{o}{=} \PY{n}{X}\PY{o}{*}\PY{o}{*}\PY{n}{i}
                      \PY{n}{X2} \PY{o}{=} \PY{n}{np}\PY{o}{.}\PY{n}{concatenate}\PY{p}{(}\PY{p}{(}\PY{n}{X2}\PY{p}{,} \PY{n}{Xadd}\PY{p}{)}\PY{p}{,} \PY{n}{axis}\PY{o}{=}\PY{l+m+mi}{1}\PY{p}{)}
              
              \PY{k}{return} \PY{n}{X2}
          
          
          \PY{c+c1}{\PYZsh{}Calcular el error del modelo de regresión logística}
          \PY{c+c1}{\PYZsh{}Si es diferente el Y\PYZus{}estimado con el Y\PYZus{}real cuenta como un error}
          \PY{k}{def} \PY{n+nf}{error\PYZus{}logistic}\PY{p}{(}\PY{n}{Y\PYZus{}lest}\PY{p}{,} \PY{n}{Y}\PY{p}{)}\PY{p}{:}
              \PY{n}{error} \PY{o}{=} \PY{l+m+mi}{0}
              \PY{k}{for} \PY{n}{ye}\PY{p}{,} \PY{n}{y} \PY{o+ow}{in} \PY{n+nb}{zip}\PY{p}{(}\PY{n}{Y\PYZus{}lest}\PY{p}{,} \PY{n}{Y}\PY{p}{)}\PY{p}{:}
                  \PY{k}{if} \PY{n}{ye} \PY{o}{!=} \PY{n}{y}\PY{p}{:}
                      \PY{n}{error} \PY{o}{+}\PY{o}{=} \PY{l+m+mi}{1}
              
              \PY{n}{error} \PY{o}{=} \PY{n}{error}\PY{o}{/}\PY{n}{np}\PY{o}{.}\PY{n}{size}\PY{p}{(}\PY{n}{Y}\PY{p}{)}
              
              \PY{c+c1}{\PYZsh{}print (\PYZdq{}La eficiencia en esta iteración fue: \PYZdq{}+str(1\PYZhy{}error)+\PYZsq{}\PYZbs{}n\PYZsq{})}
              
              \PY{k}{return} \PY{n}{error}
          
          \PY{l+s+sd}{\PYZdq{}\PYZdq{}\PYZdq{}Gradiente descendente para regresión logistica}
          \PY{l+s+sd}{X: Matriz de datos extendida.}
          \PY{l+s+sd}{W: Vector de parámetros del modelo}
          \PY{l+s+sd}{eta: Taza de aprendizaje}
          \PY{l+s+sd}{\PYZdq{}\PYZdq{}\PYZdq{}}
          \PY{k}{def} \PY{n+nf}{gradiente\PYZus{}descendente\PYZus{}logistic}\PY{p}{(}\PY{n}{X}\PY{p}{,}\PY{n}{Y}\PY{p}{,}\PY{n}{grado}\PY{p}{,}\PY{n}{eta}\PY{p}{,} \PY{n}{showPlot}\PY{o}{=}\PY{k+kc}{False}\PY{p}{)}\PY{p}{:}
              
              \PY{c+c1}{\PYZsh{}Extendemos la matriz}
              \PY{n}{unos} \PY{o}{=} \PY{n}{np}\PY{o}{.}\PY{n}{array}\PY{p}{(}\PY{p}{[}\PY{n}{np}\PY{o}{.}\PY{n}{ones}\PY{p}{(}\PY{n}{np}\PY{o}{.}\PY{n}{size}\PY{p}{(}\PY{n}{X}\PY{p}{,}\PY{l+m+mi}{0}\PY{p}{)}\PY{p}{)}\PY{p}{]}\PY{p}{)}
              \PY{c+c1}{\PYZsh{}Concatenar el vector de unos con la matriz X}
              \PY{n}{X} \PY{o}{=} \PY{n}{np}\PY{o}{.}\PY{n}{concatenate}\PY{p}{(}\PY{p}{(}\PY{n}{unos}\PY{o}{.}\PY{n}{T}\PY{p}{,} \PY{n}{X}\PY{p}{)}\PY{p}{,} \PY{n}{axis}\PY{o}{=}\PY{l+m+mi}{1}\PY{p}{)}
              \PY{n}{X} \PY{o}{=} \PY{n}{X}\PY{o}{.}\PY{n}{reshape}\PY{p}{(}\PY{n}{np}\PY{o}{.}\PY{n}{size}\PY{p}{(}\PY{n}{X}\PY{p}{,}\PY{l+m+mi}{0}\PY{p}{)}\PY{p}{,}\PY{n}{np}\PY{o}{.}\PY{n}{size}\PY{p}{(}\PY{n}{X}\PY{p}{,}\PY{l+m+mi}{1}\PY{p}{)}\PY{p}{)}
              
              \PY{n}{Y} \PY{o}{=} \PY{n}{Y}\PY{o}{.}\PY{n}{reshape}\PY{p}{(}\PY{n}{np}\PY{o}{.}\PY{n}{size}\PY{p}{(}\PY{n}{Y}\PY{p}{)}\PY{p}{,} \PY{l+m+mi}{1}\PY{p}{)}
              
              \PY{c+c1}{\PYZsh{}Tomamos el número de variables del problema}
              \PY{n}{d} \PY{o}{=} \PY{n}{np}\PY{o}{.}\PY{n}{size}\PY{p}{(}\PY{n}{X}\PY{p}{,}\PY{l+m+mi}{1}\PY{p}{)}
          
              \PY{c+c1}{\PYZsh{}Tomamos el número de muestras de la base de datos}
              \PY{n}{N} \PY{o}{=} \PY{n}{np}\PY{o}{.}\PY{n}{size}\PY{p}{(}\PY{n}{X}\PY{p}{,}\PY{l+m+mi}{0}\PY{p}{)}
              
              \PY{c+c1}{\PYZsh{}Inicializamos el vector de parámetros aleatoriamente}
              \PY{c+c1}{\PYZsh{}Want = np.random.randn(d)}
              \PY{n}{W} \PY{o}{=} \PY{n}{np}\PY{o}{.}\PY{n}{zeros}\PY{p}{(}\PY{n}{d}\PY{p}{)}
              \PY{n}{W} \PY{o}{=} \PY{n}{W}\PY{o}{.}\PY{n}{reshape}\PY{p}{(}\PY{n}{np}\PY{o}{.}\PY{n}{size}\PY{p}{(}\PY{n}{W}\PY{p}{)}\PY{p}{,}\PY{l+m+mi}{1}\PY{p}{)}
          
              \PY{n}{eta} \PY{o}{=} \PY{n}{eta}
              
              \PY{n}{iteraciones} \PY{o}{=} \PY{l+m+mi}{1000}
              \PY{n}{errores} \PY{o}{=} \PY{n}{np}\PY{o}{.}\PY{n}{zeros}\PY{p}{(}\PY{n}{iteraciones}\PY{p}{)}
              \PY{n}{K} \PY{o}{=} \PY{n}{eta}\PY{o}{/}\PY{n}{N}
              \PY{k}{for} \PY{n+nb}{iter} \PY{o+ow}{in} \PY{n+nb}{range}\PY{p}{(}\PY{n}{iteraciones}\PY{p}{)}\PY{p}{:}
          
                  \PY{n}{Y\PYZus{}estimado} \PY{o}{=} \PY{n}{logistic\PYZus{}regression}\PY{p}{(}\PY{n}{X}\PY{p}{,}\PY{n}{W}\PY{p}{)}
                  \PY{c+c1}{\PYZsh{}Error en clasificación}
                  \PY{n}{error} \PY{o}{=} \PY{n}{error\PYZus{}logistic}\PY{p}{(}\PY{n}{Y\PYZus{}estimado}\PY{p}{,}\PY{n}{Y}\PY{p}{)}
                  \PY{n}{errores}\PY{p}{[}\PY{n+nb}{iter}\PY{p}{]} \PY{o}{=} \PY{n}{error}
          
                  \PY{c+c1}{\PYZsh{}Aquí debe completar el código con la regla de actualización de los parámetros W para regresión}
                  \PY{c+c1}{\PYZsh{}logística. Tenga en cuenta los nombres de las variables ya creadas: W, X, Y        }
                  \PY{n}{loss} \PY{o}{=} \PY{n}{Y\PYZus{}estimado} \PY{o}{\PYZhy{}} \PY{n}{Y}
                  \PY{n}{W} \PY{o}{=} \PY{n}{W} \PY{o}{\PYZhy{}}\PY{n}{K}\PY{o}{*}\PY{p}{(}\PY{n}{np}\PY{o}{.}\PY{n}{dot}\PY{p}{(}\PY{n}{X}\PY{o}{.}\PY{n}{T}\PY{p}{,}\PY{n}{loss}\PY{p}{)}\PY{p}{)}
                  
                  
              
              \PY{c+c1}{\PYZsh{}Aquí debe completar el código para realizar la gráfica del error de clasificación vs. iteraciones}
              \PY{k}{if} \PY{n}{showPlot} \PY{o}{==} \PY{k+kc}{True}\PY{p}{:}
                  \PY{n}{plt}\PY{o}{.}\PY{n}{plot}\PY{p}{(}\PY{n}{errores}\PY{p}{)}
                  \PY{n}{plt}\PY{o}{.}\PY{n}{title}\PY{p}{(}\PY{l+s+s1}{\PYZsq{}}\PY{l+s+s1}{Figura 4: Error vs Iteración}\PY{l+s+s1}{\PYZsq{}}\PY{p}{)}
                  \PY{n}{plt}\PY{o}{.}\PY{n}{xlabel}\PY{p}{(}\PY{l+s+s1}{\PYZsq{}}\PY{l+s+s1}{Iteraciones}\PY{l+s+s1}{\PYZsq{}}\PY{p}{)}
                  \PY{n}{plt}\PY{o}{.}\PY{n}{ylabel}\PY{p}{(}\PY{l+s+s1}{\PYZsq{}}\PY{l+s+s1}{Errores}\PY{l+s+s1}{\PYZsq{}}\PY{p}{)}
                  \PY{n}{plt}\PY{o}{.}\PY{n}{show}\PY{p}{(}\PY{p}{)}
              
              \PY{c+c1}{\PYZsh{}print (\PYZsq{}Vector de parámetros del modelo:\PYZbs{}n\PYZsq{})}
              \PY{c+c1}{\PYZsh{}print (W)}
              \PY{c+c1}{\PYZsh{}print (\PYZsq{}\PYZbs{}nError de entrenamiento = \PYZsq{} + str(errores[\PYZhy{}1]))}
              
              
              
              \PY{k}{return} \PY{n}{W}\PY{p}{,}\PY{n}{errores}\PY{p}{[}\PY{o}{\PYZhy{}}\PY{l+m+mi}{1}\PY{p}{]}
\end{Verbatim}


    \subsection{Ejercicio 3: Entrenamiento}\label{ejercicio-3-entrenamiento}

\begin{enumerate}
\def\labelenumi{\arabic{enumi}.}
\tightlist
\item
  Complete el código de la siguiente celda llamando el método
  gradiente\_descedente\_logistic, se debe pasar los parámetros que
  corresponde con la tabla de resultados de abajo
\item
  Ejecute el entrenamiento
\item
  Llene la tabla de resultados.
\end{enumerate}

    \begin{Verbatim}[commandchars=\\\{\}]
{\color{incolor}In [{\color{incolor}128}]:} \PY{k+kn}{from} \PY{n+nn}{numpy} \PY{k}{import} \PY{n}{random}
          \PY{k+kn}{import} \PY{n+nn}{math}
          \PY{n}{N} \PY{o}{=} \PY{n}{np}\PY{o}{.}\PY{n}{size}\PY{p}{(}\PY{n}{X}\PY{p}{,}\PY{l+m+mi}{0}\PY{p}{)}
          
          \PY{c+c1}{\PYZsh{} \PYZsh{}Se modifica la matriz de datos original de acuerdo al grado del polinomio ingresado para el modelo}
          \PY{n}{grado} \PY{o}{=} \PY{l+m+mi}{3}
          \PY{n}{X2} \PY{o}{=} \PY{n}{potenciaPolinomio}\PY{p}{(}\PY{n}{X}\PY{p}{,}\PY{n}{grado}\PY{p}{)}
          
          \PY{c+c1}{\PYZsh{}Dejamos algunas muestras para el proceso de entrenamiento y otras para evaluar qué tan bueno fue el aprendizaje del modelo}
          \PY{n}{random}\PY{o}{.}\PY{n}{seed}\PY{p}{(}\PY{l+m+mi}{1}\PY{p}{)}
          \PY{n}{ind}\PY{o}{=}\PY{n}{np}\PY{o}{.}\PY{n}{random}\PY{o}{.}\PY{n}{permutation}\PY{p}{(}\PY{n}{N}\PY{p}{)}
          \PY{n}{Xtrain} \PY{o}{=} \PY{n}{X2}\PY{p}{[}\PY{n}{ind}\PY{p}{[}\PY{l+m+mi}{0}\PY{p}{:}\PY{n+nb}{int}\PY{p}{(}\PY{n}{math}\PY{o}{.}\PY{n}{ceil}\PY{p}{(}\PY{l+m+mf}{0.7}\PY{o}{*}\PY{n}{N}\PY{p}{)}\PY{p}{)}\PY{p}{]}\PY{p}{,}\PY{p}{:}\PY{p}{]}
          \PY{n}{Xtest} \PY{o}{=} \PY{n}{X2}\PY{p}{[}\PY{n}{ind}\PY{p}{[}\PY{n+nb}{int}\PY{p}{(}\PY{n}{math}\PY{o}{.}\PY{n}{ceil}\PY{p}{(}\PY{l+m+mf}{0.7}\PY{o}{*}\PY{n}{N}\PY{p}{)}\PY{p}{)}\PY{p}{:}\PY{n}{N}\PY{p}{]}\PY{p}{,}\PY{p}{:}\PY{p}{]}
          \PY{n}{Ytrain} \PY{o}{=} \PY{n}{Y}\PY{p}{[}\PY{n}{ind}\PY{p}{[}\PY{l+m+mi}{0}\PY{p}{:}\PY{n+nb}{int}\PY{p}{(}\PY{n}{math}\PY{o}{.}\PY{n}{ceil}\PY{p}{(}\PY{l+m+mf}{0.7}\PY{o}{*}\PY{n}{N}\PY{p}{)}\PY{p}{)}\PY{p}{]}\PY{p}{]}
          \PY{n}{Ytest} \PY{o}{=} \PY{n}{Y}\PY{p}{[}\PY{n}{ind}\PY{p}{[}\PY{n+nb}{int}\PY{p}{(}\PY{n}{math}\PY{o}{.}\PY{n}{ceil}\PY{p}{(}\PY{l+m+mf}{0.7}\PY{o}{*}\PY{n}{N}\PY{p}{)}\PY{p}{)}\PY{p}{:}\PY{n}{N}\PY{p}{]}\PY{p}{]}
          
          \PY{c+c1}{\PYZsh{}Normalizamos los datos}
          \PY{n}{media} \PY{o}{=} \PY{n}{np}\PY{o}{.}\PY{n}{mean}\PY{p}{(}\PY{n}{Xtrain}\PY{p}{)}
          \PY{n}{desvia} \PY{o}{=} \PY{n}{np}\PY{o}{.}\PY{n}{std}\PY{p}{(}\PY{n}{Xtrain}\PY{p}{)}
          \PY{n}{Xtrain} \PY{o}{=} \PY{n}{stats}\PY{o}{.}\PY{n}{stats}\PY{o}{.}\PY{n}{zscore}\PY{p}{(}\PY{n}{Xtrain}\PY{p}{)}
          \PY{n}{Xtest} \PY{o}{=} \PY{p}{(}\PY{n}{Xtest} \PY{o}{\PYZhy{}} \PY{n}{np}\PY{o}{.}\PY{n}{matlib}\PY{o}{.}\PY{n}{repmat}\PY{p}{(}\PY{n}{media}\PY{p}{,} \PY{n}{Xtest}\PY{o}{.}\PY{n}{shape}\PY{p}{[}\PY{l+m+mi}{0}\PY{p}{]}\PY{p}{,} \PY{l+m+mi}{1}\PY{p}{)}\PY{p}{)}\PY{o}{/}\PY{n}{np}\PY{o}{.}\PY{n}{matlib}\PY{o}{.}\PY{n}{repmat}\PY{p}{(}\PY{n}{desvia}\PY{p}{,} \PY{n}{Xtest}\PY{o}{.}\PY{n}{shape}\PY{p}{[}\PY{l+m+mi}{0}\PY{p}{]}\PY{p}{,} \PY{l+m+mi}{1}\PY{p}{)}
          
          \PY{n}{eta} \PY{o}{=} \PY{l+m+mi}{1}
          
          \PY{c+c1}{\PYZsh{}Complete la siguiente línea de código llamando el método gradiente\PYZus{}descendente con sus respectivos argumentos}
          \PY{n}{W}\PY{p}{,}\PY{n}{errorTrain} \PY{o}{=} \PY{n}{gradiente\PYZus{}descendente\PYZus{}logistic}\PY{p}{(}\PY{n}{Xtrain}\PY{p}{,}\PY{n}{Ytrain}\PY{p}{,}\PY{n}{grado}\PY{p}{,}\PY{n}{eta}\PY{p}{,} \PY{k+kc}{True}\PY{p}{)}
          
          \PY{c+c1}{\PYZsh{}Evaluamos las predicciones del modelo con los datos de test}
          \PY{n}{unos} \PY{o}{=} \PY{n}{np}\PY{o}{.}\PY{n}{array}\PY{p}{(}\PY{p}{[}\PY{n}{np}\PY{o}{.}\PY{n}{ones}\PY{p}{(}\PY{n}{np}\PY{o}{.}\PY{n}{size}\PY{p}{(}\PY{n}{Xtest}\PY{p}{,}\PY{l+m+mi}{0}\PY{p}{)}\PY{p}{)}\PY{p}{]}\PY{p}{)}
          \PY{n}{Xtest2} \PY{o}{=} \PY{n}{np}\PY{o}{.}\PY{n}{concatenate}\PY{p}{(}\PY{p}{(}\PY{n}{unos}\PY{o}{.}\PY{n}{T}\PY{p}{,} \PY{n}{Xtest}\PY{p}{)}\PY{p}{,} \PY{n}{axis}\PY{o}{=}\PY{l+m+mi}{1}\PY{p}{)}
          \PY{n}{Xtest2} \PY{o}{=} \PY{n}{Xtest2}\PY{o}{.}\PY{n}{reshape}\PY{p}{(}\PY{n}{np}\PY{o}{.}\PY{n}{size}\PY{p}{(}\PY{n}{Xtest2}\PY{p}{,}\PY{l+m+mi}{0}\PY{p}{)}\PY{p}{,}\PY{n}{np}\PY{o}{.}\PY{n}{size}\PY{p}{(}\PY{n}{Xtest2}\PY{p}{,}\PY{l+m+mi}{1}\PY{p}{)}\PY{p}{)}
          \PY{n}{Yest} \PY{o}{=} \PY{n}{logistic\PYZus{}regression}\PY{p}{(}\PY{n}{Xtest2}\PY{p}{,} \PY{n}{W}\PY{p}{,} \PY{k+kc}{True} \PY{p}{)}
          \PY{n}{Error} \PY{o}{=} \PY{n}{error\PYZus{}logistic}\PY{p}{(}\PY{n}{Yest}\PY{p}{,}\PY{n}{Ytest}\PY{p}{)}
          \PY{n+nb}{print}\PY{p}{(}\PY{l+s+s1}{\PYZsq{}}\PY{l+s+se}{\PYZbs{}n}\PY{l+s+s1}{Error durante la prueba = }\PY{l+s+s1}{\PYZsq{}} \PY{o}{+} \PY{n+nb}{str}\PY{p}{(}\PY{n}{Error}\PY{p}{)}\PY{p}{)}
\end{Verbatim}


    \begin{center}
    \adjustimage{max size={0.9\linewidth}{0.9\paperheight}}{output_20_0.png}
    \end{center}
    { \hspace*{\fill} \\}
    
    \begin{center}
    \adjustimage{max size={0.9\linewidth}{0.9\paperheight}}{output_20_1.png}
    \end{center}
    { \hspace*{\fill} \\}
    
    \begin{Verbatim}[commandchars=\\\{\}]

Error durante la prueba = 0.5

    \end{Verbatim}

    \begin{Verbatim}[commandchars=\\\{\}]
{\color{incolor}In [{\color{incolor}129}]:} \PY{k}{def} \PY{n+nf}{Training}\PY{p}{(}\PY{n}{X}\PY{p}{,}\PY{n}{Y}\PY{p}{,}\PY{n}{eta}\PY{p}{,}\PY{n}{grado}\PY{p}{)}\PY{p}{:}
              \PY{k+kn}{from} \PY{n+nn}{numpy} \PY{k}{import} \PY{n}{random}
              \PY{k+kn}{import} \PY{n+nn}{math}
              \PY{n}{N} \PY{o}{=} \PY{n}{np}\PY{o}{.}\PY{n}{size}\PY{p}{(}\PY{n}{X}\PY{p}{,}\PY{l+m+mi}{0}\PY{p}{)}
          
              \PY{c+c1}{\PYZsh{} \PYZsh{}Se modifica la matriz de datos original de acuerdo al grado del polinomio ingresado para el modelo}
              \PY{n}{grado} \PY{o}{=} \PY{n}{grado}
              \PY{n}{X2} \PY{o}{=} \PY{n}{potenciaPolinomio}\PY{p}{(}\PY{n}{X}\PY{p}{,}\PY{n}{grado}\PY{p}{)}
          
              \PY{c+c1}{\PYZsh{}Dejamos algunas muestras para el proceso de entrenamiento y otras para evaluar qué tan bueno fue el aprendizaje del modelo}
              \PY{n}{random}\PY{o}{.}\PY{n}{seed}\PY{p}{(}\PY{l+m+mi}{1}\PY{p}{)}
              \PY{n}{ind}\PY{o}{=}\PY{n}{np}\PY{o}{.}\PY{n}{random}\PY{o}{.}\PY{n}{permutation}\PY{p}{(}\PY{n}{N}\PY{p}{)}
              \PY{n}{Xtrain} \PY{o}{=} \PY{n}{X2}\PY{p}{[}\PY{n}{ind}\PY{p}{[}\PY{l+m+mi}{0}\PY{p}{:}\PY{n+nb}{int}\PY{p}{(}\PY{n}{math}\PY{o}{.}\PY{n}{ceil}\PY{p}{(}\PY{l+m+mf}{0.7}\PY{o}{*}\PY{n}{N}\PY{p}{)}\PY{p}{)}\PY{p}{]}\PY{p}{,}\PY{p}{:}\PY{p}{]}
              \PY{n}{Xtest} \PY{o}{=} \PY{n}{X2}\PY{p}{[}\PY{n}{ind}\PY{p}{[}\PY{n+nb}{int}\PY{p}{(}\PY{n}{math}\PY{o}{.}\PY{n}{ceil}\PY{p}{(}\PY{l+m+mf}{0.7}\PY{o}{*}\PY{n}{N}\PY{p}{)}\PY{p}{)}\PY{p}{:}\PY{n}{N}\PY{p}{]}\PY{p}{,}\PY{p}{:}\PY{p}{]}
              \PY{n}{Ytrain} \PY{o}{=} \PY{n}{Y}\PY{p}{[}\PY{n}{ind}\PY{p}{[}\PY{l+m+mi}{0}\PY{p}{:}\PY{n+nb}{int}\PY{p}{(}\PY{n}{math}\PY{o}{.}\PY{n}{ceil}\PY{p}{(}\PY{l+m+mf}{0.7}\PY{o}{*}\PY{n}{N}\PY{p}{)}\PY{p}{)}\PY{p}{]}\PY{p}{]}
              \PY{n}{Ytest} \PY{o}{=} \PY{n}{Y}\PY{p}{[}\PY{n}{ind}\PY{p}{[}\PY{n+nb}{int}\PY{p}{(}\PY{n}{math}\PY{o}{.}\PY{n}{ceil}\PY{p}{(}\PY{l+m+mf}{0.7}\PY{o}{*}\PY{n}{N}\PY{p}{)}\PY{p}{)}\PY{p}{:}\PY{n}{N}\PY{p}{]}\PY{p}{]}
          
              \PY{c+c1}{\PYZsh{}Normalizamos los datos}
              \PY{n}{media} \PY{o}{=} \PY{n}{np}\PY{o}{.}\PY{n}{mean}\PY{p}{(}\PY{n}{Xtrain}\PY{p}{)}
              \PY{n}{desvia} \PY{o}{=} \PY{n}{np}\PY{o}{.}\PY{n}{std}\PY{p}{(}\PY{n}{Xtrain}\PY{p}{)}
              \PY{n}{Xtrain} \PY{o}{=} \PY{n}{stats}\PY{o}{.}\PY{n}{stats}\PY{o}{.}\PY{n}{zscore}\PY{p}{(}\PY{n}{Xtrain}\PY{p}{)}
              \PY{n}{Xtest} \PY{o}{=} \PY{p}{(}\PY{n}{Xtest} \PY{o}{\PYZhy{}} \PY{n}{np}\PY{o}{.}\PY{n}{matlib}\PY{o}{.}\PY{n}{repmat}\PY{p}{(}\PY{n}{media}\PY{p}{,} \PY{n}{Xtest}\PY{o}{.}\PY{n}{shape}\PY{p}{[}\PY{l+m+mi}{0}\PY{p}{]}\PY{p}{,} \PY{l+m+mi}{1}\PY{p}{)}\PY{p}{)}\PY{o}{/}\PY{n}{np}\PY{o}{.}\PY{n}{matlib}\PY{o}{.}\PY{n}{repmat}\PY{p}{(}\PY{n}{desvia}\PY{p}{,} \PY{n}{Xtest}\PY{o}{.}\PY{n}{shape}\PY{p}{[}\PY{l+m+mi}{0}\PY{p}{]}\PY{p}{,} \PY{l+m+mi}{1}\PY{p}{)}
          
              \PY{n}{eta} \PY{o}{=} \PY{n}{eta}
          
              \PY{c+c1}{\PYZsh{}Complete la siguiente línea de código llamando el método gradiente\PYZus{}descendente con sus respectivos argumentos}
              \PY{n}{W}\PY{p}{,}\PY{n}{errorTraining} \PY{o}{=} \PY{n}{gradiente\PYZus{}descendente\PYZus{}logistic}\PY{p}{(}\PY{n}{Xtrain}\PY{p}{,}\PY{n}{Ytrain}\PY{p}{,}\PY{n}{grado}\PY{p}{,}\PY{n}{eta}\PY{p}{)}
          
              \PY{c+c1}{\PYZsh{}Evaluamos las predicciones del modelo con los datos de test}
              \PY{n}{unos} \PY{o}{=} \PY{n}{np}\PY{o}{.}\PY{n}{array}\PY{p}{(}\PY{p}{[}\PY{n}{np}\PY{o}{.}\PY{n}{ones}\PY{p}{(}\PY{n}{np}\PY{o}{.}\PY{n}{size}\PY{p}{(}\PY{n}{Xtest}\PY{p}{,}\PY{l+m+mi}{0}\PY{p}{)}\PY{p}{)}\PY{p}{]}\PY{p}{)}
              \PY{n}{Xtest2} \PY{o}{=} \PY{n}{np}\PY{o}{.}\PY{n}{concatenate}\PY{p}{(}\PY{p}{(}\PY{n}{unos}\PY{o}{.}\PY{n}{T}\PY{p}{,} \PY{n}{Xtest}\PY{p}{)}\PY{p}{,} \PY{n}{axis}\PY{o}{=}\PY{l+m+mi}{1}\PY{p}{)}
              \PY{n}{Xtest2} \PY{o}{=} \PY{n}{Xtest2}\PY{o}{.}\PY{n}{reshape}\PY{p}{(}\PY{n}{np}\PY{o}{.}\PY{n}{size}\PY{p}{(}\PY{n}{Xtest2}\PY{p}{,}\PY{l+m+mi}{0}\PY{p}{)}\PY{p}{,}\PY{n}{np}\PY{o}{.}\PY{n}{size}\PY{p}{(}\PY{n}{Xtest2}\PY{p}{,}\PY{l+m+mi}{1}\PY{p}{)}\PY{p}{)}
              \PY{n}{Yest} \PY{o}{=} \PY{n}{logistic\PYZus{}regression}\PY{p}{(}\PY{n}{Xtest2}\PY{p}{,} \PY{n}{W}\PY{p}{,} \PY{k+kc}{False}\PY{p}{)}
              \PY{n}{ErrorTest} \PY{o}{=} \PY{n}{error\PYZus{}logistic}\PY{p}{(}\PY{n}{Yest}\PY{p}{,}\PY{n}{Ytest}\PY{p}{)}
              \PY{c+c1}{\PYZsh{}print(\PYZsq{}\PYZbs{}nError durante la prueba = \PYZsq{} + str(Error))}
              \PY{k}{return} \PY{n}{W}\PY{p}{,}\PY{n}{errorTraining}\PY{p}{,}\PY{n}{ErrorTest}
\end{Verbatim}


    \paragraph{Tabla de resultados}\label{tabla-de-resultados}

    \begin{Verbatim}[commandchars=\\\{\}]
{\color{incolor}In [{\color{incolor}130}]:} \PY{k+kn}{import} \PY{n+nn}{pandas} \PY{k}{as} \PY{n+nn}{pd}
          \PY{k+kn}{import} \PY{n+nn}{qgrid}
          \PY{n}{randn} \PY{o}{=} \PY{n}{np}\PY{o}{.}\PY{n}{random}\PY{o}{.}\PY{n}{randn}
          \PY{n}{df\PYZus{}types} \PY{o}{=} \PY{n}{pd}\PY{o}{.}\PY{n}{DataFrame}\PY{p}{(}\PY{p}{\PYZob{}}
              \PY{l+s+s1}{\PYZsq{}}\PY{l+s+s1}{Tasa de aprendizaje}\PY{l+s+s1}{\PYZsq{}} \PY{p}{:} \PY{n}{pd}\PY{o}{.}\PY{n}{Series}\PY{p}{(}\PY{p}{[}\PY{l+s+s1}{\PYZsq{}}\PY{l+s+s1}{1}\PY{l+s+s1}{\PYZsq{}}\PY{p}{,} \PY{l+s+s1}{\PYZsq{}}\PY{l+s+s1}{1}\PY{l+s+s1}{\PYZsq{}}\PY{p}{,} \PY{l+s+s1}{\PYZsq{}}\PY{l+s+s1}{1}\PY{l+s+s1}{\PYZsq{}}\PY{p}{,} \PY{l+s+s1}{\PYZsq{}}\PY{l+s+s1}{1}\PY{l+s+s1}{\PYZsq{}}\PY{p}{,} \PY{l+s+s1}{\PYZsq{}}\PY{l+s+s1}{1}\PY{l+s+s1}{\PYZsq{}}\PY{p}{,} \PY{l+s+s1}{\PYZsq{}}\PY{l+s+s1}{0.1}\PY{l+s+s1}{\PYZsq{}}\PY{p}{,} \PY{l+s+s1}{\PYZsq{}}\PY{l+s+s1}{0.1}\PY{l+s+s1}{\PYZsq{}}\PY{p}{,} \PY{l+s+s1}{\PYZsq{}}\PY{l+s+s1}{0.1}\PY{l+s+s1}{\PYZsq{}}\PY{p}{,} \PY{l+s+s1}{\PYZsq{}}\PY{l+s+s1}{0.1}\PY{l+s+s1}{\PYZsq{}}\PY{p}{,} \PY{l+s+s1}{\PYZsq{}}\PY{l+s+s1}{0.1}\PY{l+s+s1}{\PYZsq{}}\PY{p}{,} \PY{l+s+s1}{\PYZsq{}}\PY{l+s+s1}{0.001}\PY{l+s+s1}{\PYZsq{}}\PY{p}{,} \PY{l+s+s1}{\PYZsq{}}\PY{l+s+s1}{0.001}\PY{l+s+s1}{\PYZsq{}}\PY{p}{,} \PY{l+s+s1}{\PYZsq{}}\PY{l+s+s1}{0.001}\PY{l+s+s1}{\PYZsq{}}\PY{p}{,} \PY{l+s+s1}{\PYZsq{}}\PY{l+s+s1}{0.001}\PY{l+s+s1}{\PYZsq{}}\PY{p}{,} \PY{l+s+s1}{\PYZsq{}}\PY{l+s+s1}{0.001}\PY{l+s+s1}{\PYZsq{}}\PY{p}{]}\PY{p}{)}\PY{p}{,}
              \PY{l+s+s1}{\PYZsq{}}\PY{l+s+s1}{Grado del polinomio}\PY{l+s+s1}{\PYZsq{}} \PY{p}{:} \PY{n}{pd}\PY{o}{.}\PY{n}{Series}\PY{p}{(}\PY{p}{[}\PY{l+m+mi}{1}\PY{p}{,}\PY{l+m+mi}{2}\PY{p}{,}\PY{l+m+mi}{3}\PY{p}{,}\PY{l+m+mi}{4}\PY{p}{,}\PY{l+m+mi}{5}\PY{p}{,}\PY{l+m+mi}{1}\PY{p}{,}\PY{l+m+mi}{2}\PY{p}{,}\PY{l+m+mi}{3}\PY{p}{,}\PY{l+m+mi}{4}\PY{p}{,}\PY{l+m+mi}{5}\PY{p}{,}\PY{l+m+mi}{1}\PY{p}{,}\PY{l+m+mi}{2}\PY{p}{,}\PY{l+m+mi}{3}\PY{p}{,}\PY{l+m+mi}{4}\PY{p}{,}\PY{l+m+mi}{5}\PY{p}{]}\PY{p}{)}\PY{p}{\PYZcb{}}\PY{p}{)}
          \PY{n}{df\PYZus{}types}\PY{p}{[}\PY{l+s+s2}{\PYZdq{}}\PY{l+s+s2}{Error\PYZus{}Entrenamiento}\PY{l+s+s2}{\PYZdq{}}\PY{p}{]} \PY{o}{=} \PY{l+s+s2}{\PYZdq{}}\PY{l+s+s2}{\PYZdq{}}
          \PY{n}{df\PYZus{}types}\PY{p}{[}\PY{l+s+s2}{\PYZdq{}}\PY{l+s+s2}{Error\PYZus{}Prueba}\PY{l+s+s2}{\PYZdq{}}\PY{p}{]} \PY{o}{=} \PY{l+s+s2}{\PYZdq{}}\PY{l+s+s2}{\PYZdq{}}
          \PY{n}{df\PYZus{}types}\PY{o}{.}\PY{n}{set\PYZus{}index}\PY{p}{(}\PY{p}{[}\PY{l+s+s1}{\PYZsq{}}\PY{l+s+s1}{Tasa de aprendizaje}\PY{l+s+s1}{\PYZsq{}}\PY{p}{,}\PY{l+s+s1}{\PYZsq{}}\PY{l+s+s1}{Grado del polinomio}\PY{l+s+s1}{\PYZsq{}}\PY{p}{]}\PY{p}{,} \PY{n}{inplace}\PY{o}{=}\PY{k+kc}{True}\PY{p}{)}
          \PY{c+c1}{\PYZsh{}df\PYZus{}types[\PYZdq{}Error\PYZus{}Entrenamiento\PYZdq{}][2] = \PYZdq{}0.0\PYZdq{}}
          \PY{c+c1}{\PYZsh{}df\PYZus{}types[\PYZdq{}Error\PYZus{}Prueba\PYZdq{}][2] = \PYZdq{}0.5\PYZdq{}}
          \PY{n}{cont} \PY{o}{=} \PY{l+m+mi}{0}
          \PY{k}{for} \PY{n}{eta}\PY{p}{,} \PY{n}{grado} \PY{o+ow}{in} \PY{n}{df\PYZus{}types}\PY{o}{.}\PY{n}{index}\PY{p}{:}
              \PY{n}{W}\PY{p}{,}\PY{n}{errorTraining}\PY{p}{,}\PY{n}{errorTest} \PY{o}{=} \PY{n}{Training}\PY{p}{(}\PY{n}{X}\PY{p}{,}\PY{n}{Y}\PY{p}{,}\PY{n+nb}{float}\PY{p}{(}\PY{n}{eta}\PY{p}{)}\PY{p}{,}\PY{n}{grado}\PY{p}{)}
              \PY{n}{df\PYZus{}types}\PY{p}{[}\PY{l+s+s2}{\PYZdq{}}\PY{l+s+s2}{Error\PYZus{}Entrenamiento}\PY{l+s+s2}{\PYZdq{}}\PY{p}{]}\PY{p}{[}\PY{n}{cont}\PY{p}{]} \PY{o}{=} \PY{n+nb}{str}\PY{p}{(}\PY{n}{errorTraining}\PY{p}{)}
              \PY{n}{df\PYZus{}types}\PY{p}{[}\PY{l+s+s2}{\PYZdq{}}\PY{l+s+s2}{Error\PYZus{}Prueba}\PY{l+s+s2}{\PYZdq{}}\PY{p}{]}\PY{p}{[}\PY{n}{cont}\PY{p}{]} \PY{o}{=} \PY{n+nb}{str}\PY{p}{(}\PY{n}{errorTest}\PY{p}{)}
              \PY{n}{cont} \PY{o}{=} \PY{n}{cont}\PY{o}{+}\PY{l+m+mi}{1}
          
          \PY{n}{df\PYZus{}types}\PY{o}{.}\PY{n}{sort\PYZus{}index}\PY{p}{(}\PY{n}{inplace}\PY{o}{=}\PY{k+kc}{True}\PY{p}{)}
          \PY{n}{qgrid\PYZus{}widget} \PY{o}{=} \PY{n}{qgrid}\PY{o}{.}\PY{n}{show\PYZus{}grid}\PY{p}{(}\PY{n}{df\PYZus{}types}\PY{p}{,} \PY{n}{show\PYZus{}toolbar}\PY{o}{=}\PY{k+kc}{False}\PY{p}{)}
          \PY{n}{qgrid\PYZus{}widget}
\end{Verbatim}


    
    \begin{verbatim}
QgridWidget(grid_options={'fullWidthRows': True, 'syncColumnCellResize': True, 'forceFitColumns': True, 'defau…
    \end{verbatim}

    
    Ejecute la siguiente instrucción para dejar guardados en el notebook los
resultados de las pruebas.

    \begin{Verbatim}[commandchars=\\\{\}]
{\color{incolor}In [{\color{incolor}131}]:} \PY{n}{qgrid\PYZus{}widget}\PY{o}{.}\PY{n}{get\PYZus{}changed\PYZus{}df}\PY{p}{(}\PY{p}{)}
\end{Verbatim}


\begin{Verbatim}[commandchars=\\\{\}]
{\color{outcolor}Out[{\color{outcolor}131}]:}                                         Error\_Entrenamiento  \textbackslash{}
          Tasa de aprendizaje Grado del polinomio                       
          0.001               1                                   0.5   
                              2                                   0.0   
                              3                                   0.0   
                              4                                   0.0   
                              5                                   0.0   
          0.1                 1                                   0.5   
                              2                                   0.0   
                              3                                   0.0   
                              4                                   0.0   
                              5                                   0.0   
          1                   1                                   0.5   
                              2                                   0.0   
                              3                                   0.0   
                              4                                   0.0   
                              5                                   0.0   
          
                                                          Error\_Prueba  
          Tasa de aprendizaje Grado del polinomio                       
          0.001               1                                    0.5  
                              2                    0.10666666666666667  
                              3                                    0.5  
                              4                                   0.12  
                              5                                    0.5  
          0.1                 1                                    0.5  
                              2                    0.10666666666666667  
                              3                                    0.5  
                              4                                   0.12  
                              5                                    0.5  
          1                   1                                    0.5  
                              2                    0.10666666666666667  
                              3                                    0.5  
                              4                                   0.12  
                              5                                    0.5  
\end{Verbatim}
            
    \subsection{Ejercicio 4}\label{ejercicio-4}

4.1 Escriba el modelo \(f(\textbf{x},\textbf{w})\), de la mejor frontera
de decisión que encontró según la tabla de resultados.

    \begin{Verbatim}[commandchars=\\\{\}]
{\color{incolor}In [{\color{incolor}138}]:} \PY{k}{def} \PY{n+nf}{model}\PY{p}{(}\PY{n}{W}\PY{p}{)}\PY{p}{:}
              \PY{n}{m} \PY{o}{=} \PY{l+s+s1}{\PYZsq{}}\PY{l+s+s1}{\PYZsq{}}
              \PY{n}{m} \PY{o}{=} \PY{n+nb}{str}\PY{p}{(}\PY{n}{W}\PY{p}{[}\PY{l+m+mi}{0}\PY{p}{]}\PY{p}{)}
              \PY{n}{cont} \PY{o}{=} \PY{l+m+mi}{1}
              \PY{k}{for} \PY{n}{w} \PY{o+ow}{in} \PY{n+nb}{range}\PY{p}{(}\PY{l+m+mi}{1}\PY{p}{,}\PY{n+nb}{len}\PY{p}{(}\PY{n}{W}\PY{p}{)}\PY{p}{)}\PY{p}{:}
                  \PY{n}{m} \PY{o}{=}  \PY{n}{m} \PY{o}{+} \PY{l+s+s1}{\PYZsq{}}\PY{l+s+s1}{ + }\PY{l+s+s1}{\PYZsq{}} \PY{o}{+} \PY{n+nb}{str}\PY{p}{(}\PY{n}{W}\PY{p}{[}\PY{n}{w}\PY{p}{]}\PY{p}{)}\PY{o}{+} \PY{l+s+s1}{\PYZsq{}}\PY{l+s+s1}{X}\PY{l+s+s1}{\PYZsq{}} \PY{o}{+} \PY{n+nb}{str}\PY{p}{(}\PY{n}{cont}\PY{p}{)}
                  \PY{n}{cont} \PY{o}{=} \PY{n}{cont} \PY{o}{+}\PY{l+m+mi}{1}  
              \PY{n}{m} \PY{o}{=} \PY{n}{m}\PY{o}{.}\PY{n}{replace}\PY{p}{(}\PY{l+s+s1}{\PYZsq{}}\PY{l+s+s1}{[}\PY{l+s+s1}{\PYZsq{}}\PY{p}{,}\PY{l+s+s1}{\PYZsq{}}\PY{l+s+s1}{(}\PY{l+s+s1}{\PYZsq{}}\PY{p}{)}
              \PY{n}{m} \PY{o}{=} \PY{n}{m}\PY{o}{.}\PY{n}{replace}\PY{p}{(}\PY{l+s+s1}{\PYZsq{}}\PY{l+s+s1}{]}\PY{l+s+s1}{\PYZsq{}}\PY{p}{,}\PY{l+s+s1}{\PYZsq{}}\PY{l+s+s1}{)}\PY{l+s+s1}{\PYZsq{}}\PY{p}{)}
              \PY{k}{return} \PY{n}{m}
\end{Verbatim}


    \begin{Verbatim}[commandchars=\\\{\}]
{\color{incolor}In [{\color{incolor}139}]:} \PY{n}{w}\PY{p}{,}\PY{n}{errorTrain}\PY{p}{,}\PY{n}{errorTest}\PY{o}{=} \PY{n}{Training}\PY{p}{(}\PY{n}{X}\PY{p}{,}\PY{n}{Y}\PY{p}{,}\PY{l+m+mi}{1}\PY{p}{,}\PY{l+m+mi}{2}\PY{p}{)}
          \PY{n+nb}{print}\PY{p}{(}\PY{n}{model}\PY{p}{(}\PY{n}{w}\PY{p}{)}\PY{p}{)}
\end{Verbatim}


    \begin{Verbatim}[commandchars=\\\{\}]
(0.03714286) + (-0.02137267)X1 + (0.02323988)X2 + (0.31396627)X3 + (0.29382284)X4

    \end{Verbatim}

    4.2 Basado en el valor del error obtenido, ¿cuántas muestras de
entrenamiento y de prueba clasifica mal el modelo? (un valor para cada
conjunto). Nota. Escriba en una celda el código con el cuál obtuvo la
respuesta.

    \begin{Verbatim}[commandchars=\\\{\}]
{\color{incolor}In [{\color{incolor}142}]:} \PY{n}{samplesErrorTrain} \PY{o}{=} \PY{n}{errorTrain}\PY{o}{*}\PY{n}{np}\PY{o}{.}\PY{n}{size}\PY{p}{(}\PY{n}{Xtrain}\PY{p}{,}\PY{l+m+mi}{0}\PY{p}{)}
          \PY{n}{samplesErrorTest} \PY{o}{=} \PY{n}{errorTest}\PY{o}{*}\PY{n}{np}\PY{o}{.}\PY{n}{size}\PY{p}{(}\PY{n}{Xtest}\PY{p}{,}\PY{l+m+mi}{0}\PY{p}{)}
          \PY{n}{samplesErrorTrain}\PY{p}{,}\PY{n}{samplesErrorTest}
\end{Verbatim}


\begin{Verbatim}[commandchars=\\\{\}]
{\color{outcolor}Out[{\color{outcolor}142}]:} (0.0, 16.0)
\end{Verbatim}
            
    \subsection{Ejercicio 5}\label{ejercicio-5}

    El siguiente código utiliza un clasificador basado en Funciones
Discriminantes Gaussianas para resolver el mismo problema de
clasificación. Ejecute el código y responda las siguientes preguntas:

    \begin{Verbatim}[commandchars=\\\{\}]
{\color{incolor}In [{\color{incolor}143}]:} \PY{k}{def} \PY{n+nf}{DistribucionGaussiana}\PY{p}{(}\PY{n}{X}\PY{p}{,}\PY{n}{Mu}\PY{p}{,}\PY{n}{Sigma}\PY{p}{)}\PY{p}{:}
              
              \PY{n}{SigmaInversa} \PY{o}{=} \PY{n}{np}\PY{o}{.}\PY{n}{linalg}\PY{o}{.}\PY{n}{inv}\PY{p}{(}\PY{n}{np}\PY{o}{.}\PY{n}{array}\PY{p}{(}\PY{n}{Sigma}\PY{p}{)}\PY{p}{)}
              \PY{n}{PrimerTermino} \PY{o}{=} \PY{p}{(}\PY{l+m+mi}{1}\PY{o}{/}\PY{p}{(}\PY{l+m+mi}{2}\PY{o}{*}\PY{n}{math}\PY{o}{.}\PY{n}{pi}\PY{o}{*}\PY{n}{math}\PY{o}{.}\PY{n}{sqrt}\PY{p}{(}\PY{n}{np}\PY{o}{.}\PY{n}{linalg}\PY{o}{.}\PY{n}{det}\PY{p}{(}\PY{n}{Sigma}\PY{p}{)}\PY{p}{)}\PY{p}{)}\PY{p}{)}
              
              \PY{n}{primerDot} \PY{o}{=} \PY{n}{np}\PY{o}{.}\PY{n}{dot}\PY{p}{(}\PY{p}{(}\PY{n}{X}\PY{o}{\PYZhy{}}\PY{n}{Mu}\PY{p}{)}\PY{p}{,}\PY{n}{SigmaInversa}\PY{p}{)}
              \PY{n}{segundoDot} \PY{o}{=} \PY{n}{np}\PY{o}{.}\PY{n}{dot}\PY{p}{(}\PY{n}{primerDot}\PY{p}{,}\PY{p}{(}\PY{n}{X}\PY{o}{\PYZhy{}}\PY{n}{Mu}\PY{p}{)}\PY{o}{.}\PY{n}{T}\PY{p}{)}
              \PY{n}{Exponencial} \PY{o}{=} \PY{n}{math}\PY{o}{.}\PY{n}{exp}\PY{p}{(}\PY{o}{\PYZhy{}}\PY{l+m+mf}{0.5}\PY{o}{*}\PY{n}{segundoDot}\PY{p}{)}
              
              \PY{n}{Probabilidad} \PY{o}{=} \PY{n}{PrimerTermino} \PY{o}{*} \PY{n}{Exponencial}
              
              \PY{k}{return} \PY{n}{Probabilidad}
          
          \PY{k}{def} \PY{n+nf}{FuncionDiscriminanteG}\PY{p}{(}\PY{n}{Xtrain}\PY{p}{,}\PY{n}{Ytrain}\PY{p}{,}\PY{n}{Xtest}\PY{p}{,}\PY{n}{tipo}\PY{p}{)}\PY{p}{:}
              
              \PY{n}{N} \PY{o}{=} \PY{n}{Xtest}\PY{o}{.}\PY{n}{shape}\PY{p}{[}\PY{l+m+mi}{0}\PY{p}{]}
              \PY{c+c1}{\PYZsh{}Estimación de medias y Covarianzas}
              \PY{n}{Mu1} \PY{o}{=} \PY{n}{np}\PY{o}{.}\PY{n}{mean}\PY{p}{(}\PY{n}{Xtrain}\PY{p}{[}\PY{p}{(}\PY{n}{Ytrain}\PY{o}{==}\PY{l+m+mi}{1}\PY{p}{)}\PY{o}{.}\PY{n}{flat}\PY{p}{,}\PY{p}{:}\PY{p}{]}\PY{p}{,} \PY{n}{axis}\PY{o}{=}\PY{l+m+mi}{0}\PY{p}{)}
              \PY{n}{Mu2} \PY{o}{=} \PY{n}{np}\PY{o}{.}\PY{n}{mean}\PY{p}{(}\PY{n}{Xtrain}\PY{p}{[}\PY{p}{(}\PY{n}{Ytrain}\PY{o}{==}\PY{l+m+mi}{0}\PY{p}{)}\PY{o}{.}\PY{n}{flat}\PY{p}{,}\PY{p}{:}\PY{p}{]}\PY{p}{,} \PY{n}{axis}\PY{o}{=}\PY{l+m+mi}{0}\PY{p}{)}
            
              \PY{n}{Sigma1} \PY{o}{=} \PY{n}{np}\PY{o}{.}\PY{n}{cov}\PY{p}{(}\PY{p}{(}\PY{n}{Xtrain}\PY{p}{[}\PY{p}{(}\PY{n}{Ytrain}\PY{o}{==}\PY{l+m+mi}{1}\PY{p}{)}\PY{o}{.}\PY{n}{flat}\PY{p}{,}\PY{p}{:}\PY{p}{]}\PY{p}{)}\PY{o}{.}\PY{n}{T}\PY{p}{)}
              \PY{n}{Sigma2} \PY{o}{=} \PY{n}{np}\PY{o}{.}\PY{n}{cov}\PY{p}{(}\PY{p}{(}\PY{n}{Xtrain}\PY{p}{[}\PY{p}{(}\PY{n}{Ytrain}\PY{o}{==}\PY{l+m+mi}{0}\PY{p}{)}\PY{o}{.}\PY{n}{flat}\PY{p}{,}\PY{p}{:}\PY{p}{]}\PY{p}{)}\PY{o}{.}\PY{n}{T}\PY{p}{)}
              
              \PY{n}{Sigma3} \PY{o}{=} \PY{p}{(}\PY{l+m+mf}{0.5}\PY{o}{*}\PY{p}{(}\PY{n}{Sigma1}\PY{o}{+}\PY{n}{Sigma2}\PY{p}{)}\PY{p}{)}
              \PY{n}{Yest} \PY{o}{=} \PY{n}{np}\PY{o}{.}\PY{n}{zeros}\PY{p}{(}\PY{n}{N}\PY{p}{)}
              \PY{n}{Tipo} \PY{o}{=} \PY{n}{tipo}
              \PY{k}{for} \PY{n}{i} \PY{o+ow}{in} \PY{n+nb}{range}\PY{p}{(}\PY{n}{N}\PY{p}{)}\PY{p}{:}
                  
                      \PY{k}{if} \PY{n}{Tipo} \PY{o}{==} \PY{l+m+mi}{0} \PY{p}{:}
                          \PY{n}{p1} \PY{o}{=} \PY{n}{DistribucionGaussiana}\PY{p}{(}\PY{n}{Xtest}\PY{p}{[}\PY{n}{i}\PY{p}{,}\PY{p}{:}\PY{p}{]}\PY{p}{,}\PY{n}{Mu1}\PY{p}{,}\PY{n}{Sigma1}\PY{p}{)}
                          \PY{n}{p2} \PY{o}{=} \PY{n}{DistribucionGaussiana}\PY{p}{(}\PY{n}{Xtest}\PY{p}{[}\PY{n}{i}\PY{p}{,}\PY{p}{:}\PY{p}{]}\PY{p}{,}\PY{n}{Mu2}\PY{p}{,}\PY{n}{Sigma2}\PY{p}{)}
                      \PY{k}{elif} \PY{n}{Tipo} \PY{o}{==} \PY{l+m+mi}{1}\PY{p}{:}
                          \PY{n}{p1} \PY{o}{=} \PY{n}{DistribucionGaussiana}\PY{p}{(}\PY{n}{Xtest}\PY{p}{[}\PY{n}{i}\PY{p}{,}\PY{p}{:}\PY{p}{]}\PY{p}{,}\PY{n}{Mu1}\PY{p}{,}\PY{n}{Sigma3}\PY{p}{)}
                          \PY{n}{p2} \PY{o}{=} \PY{n}{DistribucionGaussiana}\PY{p}{(}\PY{n}{Xtest}\PY{p}{[}\PY{n}{i}\PY{p}{,}\PY{p}{:}\PY{p}{]}\PY{p}{,}\PY{n}{Mu2}\PY{p}{,}\PY{n}{Sigma3}\PY{p}{)}
                      \PY{k}{if} \PY{n}{p1} \PY{o}{\PYZgt{}}\PY{o}{=} \PY{n}{p2}\PY{p}{:}
                          \PY{n}{Yest}\PY{p}{[}\PY{n}{i}\PY{p}{]} \PY{o}{=} \PY{l+m+mi}{1}
                      \PY{k}{else}\PY{p}{:}
                          \PY{n}{Yest}\PY{p}{[}\PY{n}{i}\PY{p}{]} \PY{o}{=} \PY{l+m+mi}{0}
                          
              \PY{k}{return} \PY{n}{Yest}
\end{Verbatim}


    \begin{Verbatim}[commandchars=\\\{\}]
{\color{incolor}In [{\color{incolor}144}]:} \PY{n}{tipo} \PY{o}{=} \PY{l+m+mi}{0} \PY{c+c1}{\PYZsh{} Frontera lineal}
          \PY{n}{Yest0} \PY{o}{=} \PY{n}{FuncionDiscriminanteG}\PY{p}{(}\PY{n}{Xtrain}\PY{p}{,}\PY{n}{Ytrain}\PY{p}{,}\PY{n}{Xtest}\PY{p}{,}\PY{n}{tipo}\PY{p}{)}
          \PY{n}{Error} \PY{o}{=} \PY{n}{error\PYZus{}logistic}\PY{p}{(}\PY{n}{Yest0}\PY{p}{,}\PY{n}{Ytest}\PY{p}{)}
          \PY{n+nb}{print}\PY{p}{(}\PY{l+s+s1}{\PYZsq{}}\PY{l+s+se}{\PYZbs{}n}\PY{l+s+s1}{Error prueba (Frontera Lineal) = }\PY{l+s+s1}{\PYZsq{}} \PY{o}{+} \PY{n+nb}{str}\PY{p}{(}\PY{n}{Error}\PY{p}{)}\PY{p}{)}
          
          
          \PY{n}{tipo} \PY{o}{=} \PY{l+m+mi}{1} \PY{c+c1}{\PYZsh{}Frontera cuadrática}
          \PY{n}{Yest1} \PY{o}{=} \PY{n}{FuncionDiscriminanteG}\PY{p}{(}\PY{n}{Xtrain}\PY{p}{,}\PY{n}{Ytrain}\PY{p}{,}\PY{n}{Xtest}\PY{p}{,}\PY{n}{tipo}\PY{p}{)}
          \PY{n}{Error} \PY{o}{=} \PY{n}{error\PYZus{}logistic}\PY{p}{(}\PY{n}{Yest1}\PY{p}{,}\PY{n}{Ytest}\PY{p}{)}
          \PY{n+nb}{print}\PY{p}{(}\PY{l+s+s1}{\PYZsq{}}\PY{l+s+se}{\PYZbs{}n}\PY{l+s+s1}{Error prueba (Frontera cuadrática) = }\PY{l+s+s1}{\PYZsq{}} \PY{o}{+} \PY{n+nb}{str}\PY{p}{(}\PY{n}{Error}\PY{p}{)}\PY{p}{)}
\end{Verbatim}


    \begin{Verbatim}[commandchars=\\\{\}]

Error prueba (Frontera Lineal) = 0.34

Error prueba (Frontera cuadrática) = 0.13333333333333333

    \end{Verbatim}

    5.1 ¿Cuál tipo de frontera proporcionó mejores resultados?:

    \textbf{R//:} Frontera cuadrática debido a que presenta menos error

    5.2 Teniendo en cuenta la forma de los datos (De acuerdo con la gráfica
hecha en el punto 2), expliqué porqué el modelo de Funciones
Discriminantes Gaussianas obtiene un buen resultado:

    \textbf{R//:} Debido a que claramente se visualiza en la figura 1 que
para poder separa las clases se requiere una frontera cuadrática.


    % Add a bibliography block to the postdoc
    
    
    
    \end{document}
